\documentclass[lecture.tex]{subfiles}

\begin{document}

\subsection{Champ electrostatique}

\begin{definition}[Source du champ]
  La source du champ est une charge électrique, ou un ensemble de charge électrique fixe dans le référentiel d'étude.
\end{definition}

\begin{definition}[Charge de détection]
  La charge de détection est une charge ponctuelle, notée $q_0$, placée en un point $M(x,y,z)$ de l'espace.
\end{definition}

\begin{definition}[Champ électrique]
  Un champ électrique est la région de l'espace où si on y place la charge de détection, la source exerce sur la charge de détection une force électrique non négligeable.
\end{definition}

\medskip

\subsection{Expression général du vecteur champ electrostatique en $M$}

\subsection{Potentiel électrostatique en $M$}

\subsection{Théorème de Gauss}

\subsection{Relation de Poisson}

\subsection{Ligne de champ dans une région vide de charge}

\subsection{Théorème de l'extremum de potentiel}

\subsection{Les différentes méthodes de détermination d'un vecteur champ}

\subsection{Continuité ou discontinuité de $\vec{E}$ et $V$}

\subsection{Analogie avec la gravitation}












\end{document}
