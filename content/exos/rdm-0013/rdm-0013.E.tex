\documentclass[lecture.tex]{subfiles}

\begin{document}

\exercice{}
%\video{https://youtu.be/blablabla}
\enonce{rdm-0013}{Traction}

On réalise un essai de traction sur une éprouvette d’un certain métal. L’aire de la section $A$ est de $39,41 cm^2$ et la base de mesure vaut $L=200mm$. Les appareils de mesure fournissent les valeurs suivantes :

\begin{center}
 \begin{tabular}{|c|c|c|c|c|c|c|c|c|c|c|c|}
 \hline
 Mesures & 1 & 2 & 3 & 4 & 5 & 6 & 7 & 8 & 9 & 10 & 11 \\
 \hline
 F [kN] & 200 & 400 & 600 & 800 & 1000 & 1100 & 1200 & 1250 & 1300 & 1350 & 1380 \\
 \hline
 U (mm) & 0,142 & 0,28 & 0,422 & 0,661 & 0,702 & 0,771 & 0,864 & 0,948 & 1,2 & 1,71 & 2,501 \\
 \hline
\end{tabular}
\end{center}

\begin{enumerate}
  \item Tracer le diagramme $(\sigma,\epsilon)$ du matériau
  \item En déduire la valeur du module d’élasticité $E$.
  \item En déduire la valeur de la limite d’élasticité conventionnelle
  \item De quel métal s’agit-il probablement ? (Voir le tableau ci-dessous)
\end{enumerate}

\begin{center}
\begin{tabular}{|l|l|}
  \hline
  Matériau & Module de Young [MPa] \\
  \hline
  Acier & 205000 \\
  \hline
  Aluminium & 70000 \\
  \hline
  Fonte & 100000 \\
  \hline
  Plomb & 17000 \\
  \hline
  Titane & 110000 \\
  \hline
  Verre & 2500 \\
  \hline
\end{tabular}
\end{center}


\finenonce{rdm-0013}
\finexercice


\end{document}
