\documentclass[lecture.tex]{subfiles}

\begin{document}

\exercice{}
%\video{https://youtu.be/blablabla}
\enonce{rdm-0023}{Statique et efforts internes}

Nous considérons une poutre de longueur $4L$, en appui simple en $B$ (ponctuelle) et en appui articulé en $D$ (pivot). La poutre est soumise à deux forces ponctuelles $\vec{F}=-F \vec{y}$ en $A$ et $E$, et une force répartie entre $C$ et $D$. Le poids de la structure n'est pas pris en compte.

\medskip

On donne

\begin{center}
  \begin{tabular}{|l|l|l|l|l|l|}
    \hline
    L = 400 mm & F = 800 N & p = 1200 N/m \\
    \hline
  \end{tabular}
\end{center}

\bigskip

\begin{center}
\begin{tikzpicture}[scale=1]
\coordinate(A) at (0,0);
\coordinate(B) at (2,0);
\coordinate(C) at (4,0);
\coordinate(D) at (6,0);
\coordinate(E) at (8,0);
\draw[ultra thick,anchor=south west,cerulean]
(A) node[anchor=south west]{A}
--(B) node[anchor=south]{B}
--(C) node[anchor=south east]{C}
--(D) node[anchor=south west]{D}
--(E) node[anchor=south west]{E};
\filldraw (A) circle (1pt);
\filldraw (B) circle (1pt);
\filldraw (C) circle (1pt);
\filldraw (D) circle (1pt);
\filldraw (E) circle (1pt);
\pic at (B) {ponctuelle};
\chargecont[asparagus,thick]{C}{2}{$p$};
\draw[very thick,red,latex-] (A) --++(0,1) node[anchor=east]{$F$};
\pic at (D) {pivot};
\draw[very thick,red,latex-] (E) --++(0,1) node[anchor=east]{$F$};
\draw [thick,<->] (0,-1.5) -- (2,-1.5);
\filldraw (1,-1.5) circle (0pt) node[anchor=north] {$L$};
\draw [thick,<->] (2,-1.5) -- (4,-1.5);
\filldraw (3,-1.5) circle (0pt) node[anchor=north] {$L$};
\draw [thick,<->] (4,-1.5) -- (6,-1.5);
\filldraw (5,-1.5) circle (0pt) node[anchor=north] {$L$};
\draw [thick,<->] (6,-1.5) -- (8,-1.5);
\filldraw (7,-1.5) circle (0pt) node[anchor=north] {$L$};
%% REPERE
\draw [thick,<->] (11,0) -- (10,0) -- (10,1);
\filldraw (11,0) circle (0pt) node[anchor=north] {$x$};
\filldraw (10,1) circle (0pt) node[anchor=west] {$y$};
\filldraw (9.8,-0.2) circle (1pt);
\filldraw (9.7,-0.2) circle (0pt) node[anchor=east] {$z$};
\draw (9.8,-0.2) circle (0.15cm);
\draw [->] ($(10,0.1)+(0.4,0)$) arc (0:80:0.4);
\filldraw (10.3,0.4) circle (0pt) node[anchor=west] {${\small +}$};
\end{tikzpicture}
\end{center}

\begin{enumerate}
  \item Étudiez l’hyperstaticité du système.
  \item Etude statique.
  \begin{enumerate}
  \item Trouver les inconnus de liaison de la structure.
  \item Faire l’application numérique.
  \end{enumerate}
  \item Efforts internes.
  \begin{enumerate}
    \item Combien de coupe faut-il pour étudier les efforts internes du système ?
    \item Calculez analytiquement les expressions des efforts internes pour $x \in[0 ; 5L]$.
    \item Tracez les graphes de ces efforts internes pour $x \in[0 ; 5L]$ en précisant des valeurs numériques sur les axes.
  \end{enumerate}
\end{enumerate}

\finenonce{rdm-0023}
\finexercice

\end{document}
