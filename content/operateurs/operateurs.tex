\documentclass[lecture.tex]{subfiles}

\begin{document}

\subsection{Générlités sur les champs}

Soit un point mobile $M$ de l'espace de coordonnées cartésiennes $(x,y,z)$. Le point $M$ appartient à un champ si les propriétés locales de l'espace dépendent de la position du point $M$, donc des coordonnées $(x,y,z)$. Ces propriétés dépendent aussi du temps dans le cas général. Dans les cas particuliers où ces propriétés ne dépendent pas du temps, le champs est qualifié de "statique" ou "permanent" ou "stationnaire".

\subsection{Opérateur vectoriel nabla}

\subsection{Vecteur gradient d'un champ scalaire}

\subsection{Propriétés fondamentales du gradient}

\subsection{Expression du gradient en coordonnées cyclindro-polaires}

\subsection{Expression du gradient en coordonnées sphériques}

\subsection{Divergence d'un champ vectoriel}

\subsection{Propriété fondamentale de la divergence d'un vecteur en $M$}

\subsection{Théorème d'Ostrograski}

\subsection{Rotationnel d'un champ vectoriel}

\subsection{Propriété fondamentale du rotationnele d'un vecteur en $M$}

\subsection{Théorème de Stokes}

\subsection{Laplacien d'un champ scalaire}

\subsection{Laplacien vectoriel d'un champ vectoriel}

\subsection{Champ à circulation conservative}

\subsection{Champ à flux conservatif}

\subsection{Composition d'opérateurs}

\subsection{Opérateurs appliqués à des produits}

\subsection{Expressions en coordonnées cylindro-polaires}

\subsection{Expressions en coordonnées sphériques}
















\end{document}
